% ************************** Thesis Abstract *****************************
\begin{abstract}
%In real-time and safety-critical systems, the move towards multi-cores is becoming unavoidable in order to satisfy the increasing processing requirements and to meet the high integration trend while maintaining a reasonable power consumption. However, standard multi-core systems are mainly designed to increase average performance, whereas embedded systems have additional requirements with respect to safety, reliability and real-time behavior. Therefore, the shift to multi-cores raises several challenges the embedded systems community has to face. These challenges involve the design of certifiable multi-core platforms, the management of shared resources and the development/integration of parallel software. New issues are encountered at different steps of application development, from modeling and design to software implementation and hardware deployment. Therefore, both mul- ticore/semiconductor manufacturers and the real-time community have to meet the challenges imposed by multicores. The goal of this paper is to trigger such a discussion as an attempt to bridge the gap between the two worlds and to raise awareness about the hurdles and challenges that need to be tackled.

%These architectures are challenging for safety critical applications because they are in general not predictable,

%The difficulties increase when the multi-core hosts several applications and in particular mixed critical applications.

In real-time and safety-critical systems, the move towards multi-cores is becoming unavoidable to satisfy the increasing processing requirements while maintaining a reasonable power consumption. A common trend in real-time safety-critical embedded systems is to integrate multiple applications on a single platform. Such systems are known as mixed-criticality systems as the applications are usually characterized by different criticality levels. However, multi-core systems are mainly designed to increase average performance, whereas embedded systems, and in particular mixed-criticality system, have additional requirements on safety, reliability and real-time behavior. Therefore, the shift to multi-cores raises several challenges. These architectures are challenging for safety-critical applications because they are in general not predictable. The difficulties increase when the multi-core hosts several applications and in particular mixed-critical applications.
\par  Mixed-criticality embedded systems are gaining considerable interest, but there is a lack of model-based tools for their development. This thesis proposes a model-based approach to handle the design complexity with the support of optimization techniques and code generation methods.

\end{abstract}
